\documentclass[10pt,a4paper]{article}
\usepackage[utf8]{inputenc}
\usepackage{amsmath}
\usepackage{amsfonts}
\usepackage{amssymb}
\usepackage{graphicx}
%\usepackage{dvipng}
\usepackage{lipsum}
\usepackage[left=2cm,right=2cm,top=2cm,bottom=2cm]{geometry}


\author{Rosalie}
\title{GitHub}
\begin{document}
\maketitle
Github ne modifie pas la façon dont on travaille. On modifie notre document comme d'habitude, et on l'enregistre bien sur dans le directory qu'on a déclaré.
Puis en allant sur Github ou sur Github dekstop, des uncommitted changes apparaissent si on a effectué des modifications.

Lorsqu'on synchronise des fichiers type LaTeX, il y a plein de fichiers qui se créent et on n'en veut pas. On peut créer un fichier .gitignore, à l'intérieur duquel on met toutes les extensions à ignorer (par exemple les .aux, .bbl, ...). Ca évite à chaque synchronisation d'avoir tous les fichiers qui apparaissent.
\end{document}