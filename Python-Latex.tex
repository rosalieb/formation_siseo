\documentclass[10pt,a4paper]{article}
\usepackage[utf8]{inputenc}
\usepackage{amsmath}
\usepackage{amsfonts}
\usepackage{amssymb}
\usepackage{graphicx}
%\usepackage{dvipng}
\usepackage{lipsum}
\usepackage[left=2cm,right=2cm,top=2cm,bottom=2cm]{geometry}


\author{Rosalie Bruel}
\title{\LaTeX \& Python pour de belles figures}
\begin{document}
\maketitle
On peut utiliser Python pour faire des figures. On passe par Anaconda, puis Spyder.\\
\begin{figure}
\begin{center}
\includegraphics[width=1.\textwidth]{Anaconda}
\caption{Interface Anaconda}
\end{center}
\end{figure}

\begin{figure}
\begin{center}
\includegraphics[width=1.\textwidth]{Python}
\caption{Interface Python}
\end{center}
\end{figure}

Il faut lui charger une librairie pour qu'il puisse faire des statistiques (numpy qui est le clone de matlab par exemple). Une commande type ressemble à :
\begin{center}
\emph{import matplotlib.pyplot as plt \# Librairie graphique}
\end{center}
Lorsque l'on veut écrire une fonction, on commence par def, puis on nomme la fonction. Python reconnait qu'on est dans une fonction grâce aux espaces avant les lignes de codes. En gros tout est aligné avec un nombre d'espaces commun. On peut inclure une doc à la fonction, on la mettra entre guillemets. On retrouve l'aide en faisant help(nom de la fonction).\\
L'un des problèmes avec les figures est lorsqu'on veut avoir les légendes à la bonne taille de police (similaire à celle du corps de texte). On peut :
\begin{itemize}
\item Créer la figure directement avec les bonnes taille de police et ne pas la redimensionner.
\item Redimensionner dans \LaTeX, plus compliqué, mais plus pratique pour ajuster. La librairie matplotlib permet de travailler les figures crées en code Python directement depuis \LaTeX.
\end{itemize}



\lipsum[1-8]

\begin{figure}[htb]
\begin{center}
\includegraphics[width=0.8\textwidth]{oscillateur}
\caption{Jolie figure sur Python \ldots}
\end{center}
\end{figure}

\lipsum[1-8]

\begin{figure}[htb]
\begin{center}
\includegraphics[width=0.8\textwidth]{oscillateur-resized}
\caption{Jolie figure sur Python \ldots avec dimension définie dans le code Anaconda}
\end{center}
\end{figure}
\end{document}